%
%
%
%
\newpage
\textbf{\Huge{Problemas}}

\section{Conversa com Francisco}
\begin{enumerate}
\item Como determinar o consumo de cada operador?
\item definir explicitamente algoritmo para determinar como migrar o processamento de um operador para nós distintos
\item Como fazer o deploy de uma EPN inteira ( todos os operadores de uma aplicação serão definidos de uma só vez ou dinamicamente)
%\item Algoritmo de migração de estados (bom)
\item Listas quais são os eventos primitivos e derivados possíveis de se gerar a partir do simulador (com trânsito) com a ideia de ter mais de uma aplicação (EPN), consumindo eventos interconectados, por exemplo , para controlar semáforos, desvio de rotas.
\item Algoritmos de contração /expansão de nós para a distribuição do processamento
\item Considerar deixar de lado balanceamento de carga sobre um único operador.
\end{enumerate}

\begin{enumerate}
\item Sobre Escopo: achou ousado demais visar escalabilidade e disponibilidade (tolerancia a falhas)
\item detectou 3 dimensão : esturtura do deployment distribuindo em nuvem
\item sugeriu a escolha de 1 dos 3 aspectos para dar enfase
\item Escalabilidade -> transferencia de estado é chave nessa questão. Francisco acha que essa discussão foi perdida quando a proposta foi apresentada. Faltou abordar isso com relação aos operadores de CEP
\item Sugeriu procurar refêrencias sobre "elasticidade" em CEP, pois esses trabalhos tratam da questão do estado.
\item abordagens para torelancia a falhas: 
\begin{enumerate}
\item replay dos eventos
\item checkpoints
\item replicação completa
\end{enumerate}
\item Seria melhor restringir os operadores que serão suportados na ferramenta
\item Sobre a aplicação de exemplo: Francisco observou que tem operadores com janelas de tempo enormes. No estudo de caso, faltou detalhar a regra exata associada a cada evento.
\item Francisco perguntou a que se refere o balanceamento de carga, preciso esclarecer o modelo, o grafo do deploy
\item Deixar claro que um mesmo CEP-Worker pode processar mais de um operador
\item Francisco sugeriu que poderia haver mais de um worker no mesmo nó
\item Francisco sugeriu que um outro possivel foco do trabalho poderia ser na definição do reaproveitamento do processamento no deploy
\end{enumerate}










\section{Anotações da Fernanda sobre a Quali}
\subsection{Francisco}
\begin{enumerate}
\item Escopo: disponibilidade e escalabilidade, alta complexidade. Na proposta parece que acrescentou o deployment além do que foi citado anteriormente.
\item Sobre transferencia de estado, importante discutir sobre isso. Sobre reconstrução de estado
\item Buscar por elasticidade em CEP
\item Ele falou sobre a abordagem para tolerância a falhas, para refletir qual será usada.
\item Comentou sobre o deployment, que traz complexidades também
\item Escolher menos requisitos para resolver no mestrado
\item Discutir mais sobre os operadores
\item Para balanceamento de carga: particionamento geográfico, por exemplo.
\item Detalhar mais o modelo de balanceamento
\item Deixar claro que mais de um operador está sendo processado pelo mesmo CEP Worker
\item Ele disse que cada aplicação tem sua propria EPN
\item Comentou que talvez seja melhor focar no deployment e em reaproveitar os operadores. 
\end{enumerate}

\subsection{Fábio}
\begin{enumerate}
\item Ele concordou sobre o tamanho do escopo, muito grande
\item Para tomar cuidade com algumas afirmações, como citar gartner. Tercuidade na precisão em artigos, quando for publicar.
\item Comentou que achou estranho usar o termo "Arquitetura coreografada". Substituir por "Arquitetura baseada em coreografia"
\item "Este" trabalho é o seu trabalho, deixar claro se é o seu ou de outra pessoa
\item Ele sentiu falta de exemplos na introdução sobre cidades inteligentes para motivar
\item Falou que gostou da seção de conceitos. Só achou que na pag 10 ficou um pouco vago.
\item Tirar a menção a empresas publicas e privadas
\item Nos trabalhos relacionados engatizar pontos que se trabalho tem, mas falta nos relacionados
\item Ver se é caso de estudo ou estudo de caso
\item Notação usada nos disgramas, talvez usar um formato padrão ou explicar a notação usada
\item Explicar melhor como é o algoritmo, quando a instancia sabe que precisa ser autodestruida.
\item Arrumar o "reencia os dados para o CEP-manager" de forma mais clara
\item detalhe do texto na pagina 27 sobre a citação do arthur. Artigo não cria plataforma.
\item Evitar frases de 5 linhas.
\item Sugestão para aplicação de transporte. Lelão entre diferentes serviços como uber, lyft e SPtaxi, para competição de preço
\item cronograma apertado
\item detalhe das referencias: eliminar informações desnecessárias como DOI e ISBN
\item Versão antiga do Santana et al
\item tirar sites das referências 
\end{enumerate}

\section{Minhas anotações da qualificação}
\begin{enumerate}
\item Escopo muito grande. Escolher Escalabilidade ou Disponibilidade.
\item Rever operadores que terão suporte
\item maior foco em elasticidade em CEP
\item Técnica - cache de eventos para disponibilidade
\item (Alguns detalhes no texto estão sendo alterados diretamente)
\item discrever \cite{Diniz:2015:RAC:2814058.2814074} e \cite{7916235} como exemplo do uso de CEP em cidades inteligentes.
\item Usar "Arquitetura baseada em coreografia" ao invés de "Arquitetura Coreografada".
\item revisar o texto para não usar "Este trabalho ", mas "Em nosso trabalho ou "nesta proposta de metrado".
\item focar nas EPNs
\item reposicionar o problema
\item Colocar exemplos do estudo de caso mais cedo para engajar o leitor
\item colocar exemplos em todos os operadores e diferentes características
\item Tolerancia a falhas: explicitar que são das ferramentas
\item Mostrar a contribuição antes dos trabalhos relacionados para apontar neles como eles não resolvem o problema
\item Usar padrões de figuras para represenatação do sistema e dos Eventos(da EPN)( ou explicar tudo no texto)
\item mostrar a EPN de detecção dos eventos inteira.
\item falar exatamente os operadores usados
%\item Trocar a consulta a forma de cobrança distinta por consulta aos aplicativos ja existentes (uber, SPtaxi, lyft, 99, etc)
\item Retirar dados em excesso de cada referência.
\end{enumerate}

%\section{Anotações da Kelly}
%\subsection{Apresentação}
%\begin{enumerate}
%\item Faltou exemplos na definição de conceitos 
%\item Ficou olhando só para o Francisco 
%\item Falou rápido demais 
%\item Exemplifica mais ou menos
%\item Não falou direito sobre o InterSCity 
%\item Falou Arthur Santana
%\end{enumerate}

\subsection{Francisco}




\subsection{Fábio}
\begin{enumerate}
\item Reforçou que o trabalho está ambicioso, sugeriu focar na escalabilidade
\item sobre o texto: Cuidado com o rigor; não citar Gartner (na precisão do número de dispositivos Iot)
\item Não gostou de "Arquitetura coreografada"-> trocar por "arquitetura baseada em coreografia"
\item Pronome demonstrativo "Este" confunde o leitor. Deixar mais claro o contexto, quando a frase se referir ao seu trabalho.
\item Na introdução, faltou citar alguns exemplos de uso de CEP.
\item Os exemplos nas seções 2 e 3 são muito simples. O exemplo interessante apareceu só no final do texto.
\item Fabio elogiou a seção 2, mas faltou exemplos para ilustrar os conceitos
\item Na definição de cidades inteligentes, não citar a distinção entre empresas publicas e privadas
\item Nos trabalhos relacionados, faltou ressaltar as deficiencias e destacar as coisas que a sua pesquisa vai solucionar
\item Fabio comentou que só teve uma ideia da proposta quando chegou na seção 4
\item Mudar Caso de Estudo para Estudo de Caso
\item Faltou definir a notação dos diagramas ou usar um formato padrão. Os elementos graficos tem que ter significado
\item A parte da autodestruição do worker não está descrita de forma clara; é preciso definir isso de forma algoritma
\item Na pag 27, a citação à plataforma ficou estranha. Basta falar sobre a plataforma e citar o paper. Não precisa mencionar o nome do desenvolvedor principal
\item Fábio sugeriu outra ideia de aplicativo: leilão de corridas entre taxi, uber, 99, etc. -> avaliar viabilidade.
\item Cronograma apertado, será preciso focar no escopo 
\item Referencias estão muito sujas. DOI, ISBN, URLs não são necessários
\item Fábio sugeriu colocar URLs como notas de rodapé ou no corpo do texto. Bibliografia deve ficar para papers, articles, white papers, etc.
\end{enumerate}

\section{Alfredo}

\subsection{enviados}
\begin{enumerate}

\item Texto: O primeiro capítulo possui muitos erros, os outros menos. Tente na medida do possível evitar termos informais como "lidar com" e "levar em conta"


\item Se desde o começo você tivesse um caso de uso que quer resolver o texto fluiria melhor. O caminho natural seria mostrar os problemas de escalabilidade da abordagem orquestrada, para aí sim, buscar uma solução. Logo, na sua qualificação eu esperava ter visto esse caso de uso implementado e uma situação em que ele não funciona. Parece que você está atrasado...


\item Não me parece que arquitetutras coreografadas possuam menos acoplamento do que as orquestradas. - pág 2 Comente.

\item Acho que faltou entrar em mais detalhes na Seção 2.1.5 - ferramentas para CEP em código aberto

\item Faltou um pouco de contextualização no capítulo de trabalhos relacionados
\item Faltou explicar bem melhor o capítulo de trabalhos relacionados e usá-lo para embasar a sua abordagem. Não basta descrever, é importante apontar as limitações e qualidades
de cada trabalho. Uma tabela ao final também ajudaria.

\item Como você encontrou os trabalhos relacionados?

\item Seção 4.1.1 Por que se auto-destruir?

\item Se há vários CEP-Manager, eles usam o mesmo BD? Isso pode ir contra as boas práticas de microsserviços

\item Caso de uso: Faça primeiro o básico, só sofistique depois (avaliação), além do que as ideias propostas parecem arbitrárias (sistema de pontuação)


\item Qual o seu conhecimento do InterSCSimulator?
\end{enumerate}

\subsection{No pdf}
\begin{enumerate}
\item Camel Case nos titulos das seções
\item frases muito longas na introdução
\end{enumerate}

\section{Fábio. Comentários no pdf}

\begin{enumerate}
\item Resumo: Bom
\item (alguns itens foram corrigidos diretamente no texto ou comentados diretamente)
\item Trabalhos relacionados: E como entra seu trabalho? ficou faltando
\item \ref{CEP-Processor} Fabio: Qual o significado das flechas? Qual o significado das cores? Qual o significado dos diferentes tipos de caixas?

\item \ref{sec:tools} Fábio: Você falou duas vezes nesta frase sobre serviços públicos e privado sendo que não é tão importante assim.
\item gostou do cenário do estudo de caso
\item Cronograma: Parece apertado. Talvez demore um pouco mais...
\item Remover ISBNs, doi e URl dos artigos
\item "Muitas URLs na bibliografia. Fica Feio. Coloque as Urls no corpo do texto ou em notas de rodapé.
\end{enumerate}

\section{Resumo}

\subsection{Problemas Conceituais}

\begin{itemize}
\item Introdução: Apresentar melhor o problema e a proposta na introdução, de forma a realmente mostrar o que será resolvido e para que nos Trabalhos Relacionados exista um ponto de referencia para comparação
\item Conceitos: Criar um novo estudo de caso, com duas EPNs integradas, e utilizá-lo de exemplo na descrição dos operadores
\item Proposta: Escrever Precisamente o algoritmo de criação de novas instancias e remoção.
\item Proposta: 
\end{itemize}




\subsection{Problemas de Texto}
\begin{itemize}
\item Camel Case nos titulos das seções
\item Frases muito longas
\end{itemize}

\subsubsection{Introdução}
Reescrever quase tudo levando em conta:
\begin{itemize}
\item Frases muito longas
\item afirmações muito fortes
%\item não citar Gartner
\item evitar termos informais como "lidar com" e "levar em conta"
\end{itemize}
\subsubsection{Conceitos}
\begin{itemize}
\item 
\item todos os operadores devem ser exemplificados, com exemplos relacionados ao estudo de caso.
\end{itemize}



\subsubsection{Trabalhos Relacionados}
Apresentar melhor a proposta antes de comparar com trabalhos relacionados para mostrar como eles não são suficientes para solucionar o problema
\subsubsection{Proposta}

\begin{itemize}
\item Retirar Disponibilidade como requisito principal, ressaltando só quando a torelancia a falhas já for provida pelas ferramentas utilizadas.(5.2.1.1 ; 5.2.2.1 ; 5.3.1 ; 5.4.2.1 ; 5.2.1.4 ; 5.3.4 ; 5.3.13 ; 5.4.2.6)
\item Reconstrução do estado só será possível caso o timestamp considerado venha na própria medida. Desta forma é possível replicar o estado da seguinte forma:
\begin{enumerate}
\item Criar o operador em outro ambiente
\item repassar os mesmos eventos de entrada
\item Os eventos detectados pela segunda instancia não são enviados para o sistema.
\item a segunda instancia monitora os eventos gerados pela primeira até que os eventos gerados são os mesmos
\item quando isto acontecer, um timestamp é determinado para que a partir dele, os eventos seriam enviados pela nova instancia
\end{enumerate}
(5.2.1.2)

\item Procurar mais referencias a partir de "CEP elasticity" ou "Complex Event Processing Elasticity" (5.2.1.3; 5.3.3 ; 5.4.2.5)

\item O deployment será complicado ( 5.2.1.5 ; 5.2.1.12 ; 5.4.2.9)
\item O Estudo de Caso (5.2.2.9; 5.4.3.11) será refeito, levando em conta duas EPNs distintas de duas aplicações (5.1.5) e descrevendo seus operadores de forma explicíta ( 5.2.1.11 ; 5.3.9 ; 5.3.10 ; 5.3.16 ; 5.3.17). Uma das aplicações pode ser a já proposta, com a alteração na consulta de cobrança e dados do veículo para aplicativos já existentes (5.2.2.15 ; 5.3.18 ; 5.4.3.15) e outra sobre detecção de velocidades e acidentes, de forma que parte de cada uma usaria eventos da outra.
\item Explicar cada parte da figura e procurar usar BPMN ou UML para representações.


\end{itemize}
